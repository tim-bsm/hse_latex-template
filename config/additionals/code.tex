% !TEX root = ../../document/document.tex
% LTeX: enabled=false


%%%%%%%%%% Settings %%%%%%%%%%

% Rename listings for \autoref to work
\renewcommand\lstlistingname{\langcode}
\renewcommand\lstlistlistingname{\langcodelist}
\def\lstlistingautorefname{\langcode}

% Define own languages for listings
\newcommand{\ownlstlanguages}{
    % Define colors:
    \newcommand{\ownStyleText}{\color{black}}
    \newcommand{\ownStyleKeyword}{\color{blue}}
    \newcommand{\ownStyleString}{\color{Orchid}}
    \newcommand{\ownStyleComment}{\color{gray}\ttfamily}


    \lstdefinelanguage{dockerfile}{
        keywords={FROM, RUN, COPY, ADD, ENTRYPOINT, CMD,  ENV, ARG, WORKDIR, EXPOSE, LABEL, USER, VOLUME, STOPSIGNAL, ONBUILD, MAINTAINER, HEALTHCHECK},
        keywordstyle=\bfseries\ownStyleKeyword,
        identifierstyle=\ownStyleText,
        sensitive=false,
        comment=[l]{\#},
        commentstyle=\ownStyleComment,
        stringstyle=\ttfamily\ownStyleString,
        morestring=[b]',
        morestring=[b]"
    }

    \lstdefinelanguage{docker-compose}{
        keywords={image, environment, ports, container_name, ports, volumes, links},
        keywordstyle=\bfseries\ownStyleKeyword,
        identifierstyle=\ownStyleText,
        sensitive=false,
        comment=[l]{\#},
        commentstyle=\ownStyleComment,
        stringstyle=\ttfamily\ownStyleString,
        morestring=[b]',
        morestring=[b]"
    }

	\lstdefinelanguage{ini}{
		columns=fullflexible,
		commentstyle=\ownStyleComment,
		morecomment=[s][\ownStyleString\bfseries]{[}{]},
		morecomment=[l]{\#},
		morecomment=[l]{;},
		morekeywords={},
		otherkeywords={=},
		% otherkeywords={:},
		keywordstyle={\ownStyleKeyword\bfseries}
	}

    \lstdefinelanguage{JavaScript}{
		keywords={typeof, new, true, false, catch, function, return, null, catch, switch, var, if, in, while, do, else, case, break},
		ndkeywords={class, export, boolean, throw, implements, import, this},
		sensitive=false,
		comment=[l]{//},
		morecomment=[s]{/*}{*/},
		morestring=[b]',
		morestring=[b]"
	}

    \newcommand{\JSONliteratestyle}{\bfseries\ownStyleString}
	\lstdefinelanguage{JSON}{
		commentstyle=\ownStyleComment,
		stringstyle=\ownStyleText,
		string=[s]{"}{"},
		comment=[l]{//},
		morecomment=[s]{/*}{*/},
		delim=[l][\ownStyleKeyword]{:\ "},
		moredelim=[l][\ownStyleKeyword]{:"},
		identifierstyle=\ownStyleKeyword,
		literate=
			*{0}{{{\JSONliteratestyle 0}}}{1}
			{1}{{{\JSONliteratestyle 1}}}{1}
			{2}{{{\JSONliteratestyle 2}}}{1}
			{3}{{{\JSONliteratestyle 3}}}{1}
			{4}{{{\JSONliteratestyle 4}}}{1}
			{5}{{{\JSONliteratestyle 5}}}{1}
			{6}{{{\JSONliteratestyle 6}}}{1}
			{7}{{{\JSONliteratestyle 7}}}{1}
			{8}{{{\JSONliteratestyle 8}}}{1}
			{9}{{{\JSONliteratestyle 9}}}{1}
	}
	
    \newcommand\XMLdelimstyle{\bfseries\ownStyleString}
	\lstdefinelanguage{XML}{
		columns=fullflexible,
		commentstyle=\ownStyleComment,
		stringstyle=\ownStyleKeyword,
		morestring=[b]",
		moredelim=[s][\XMLdelimstyle]{<}{\ },
		moredelim=[s][\XMLdelimstyle]{</}{>},
		moredelim=[l][\XMLdelimstyle]{/>},
		moredelim=[l][\XMLdelimstyle]{>},
		morecomment=[s]{<?}{?>},
		morecomment=[s]{<!--}{-->},
		identifierstyle=\ownStyleText
	}

    % https://github.com/GothenburgBitFactory/guides/blob/master/20151107_de_openrheinruhr/yaml_syntax_highlighting.tex
    \newcommand\YAMLvaluestyle{\ownStyleKeyword}
    \lstdefinelanguage{YAML}{
        keywords={true,false,null,y,n},
        keywordstyle=\color{darkgray},%\bfseries,
        sensitive=false,
        comment=[l]{\#},
        morecomment=[s]{/*}{*/},
        commentstyle=\ownStyleComment,
        stringstyle=\YAMLvaluestyle,
        moredelim=[l][\color{orange}]{\&},
        moredelim=[l][\color{magenta}]{*},
        moredelim=**[il][\ownStyleText{:}\YAMLvaluestyle]{:},   % switch to value style at :
        morestring=[b]',
        morestring=[b]",
        literate =    {---}{{\ProcessThreeDashes}}3
                        {>}{{\textcolor{red}\textgreater}}1     
                        {|}{{\textcolor{red}\textbar}}1 
                        {\ -\ }{{ -\ }}3,
    }
}

% Define the default style of the listings
\lstset{
	numbers=left,						% Line numbers left
	stepnumber=1,						% Line numbers on every line
	breaklines=true,					% Break long lines, if needed
	breakautoindent=true,				% Autoindent broken lines
	postbreak=\space,					% Break at a space
	tabsize=1,							% Tabsize
	basicstyle=\ttfamily\footnotesize, 	% Font and size to small
	showspaces=false,					% Do not show spaces
	showstringspaces=false,				% Do not show spaces in strings ('')
	extendedchars=true,					% Show all characters from Latin1
	captionpos=b,						% Sets the caption-position to bottom
	xleftmargin=0pt,
	xrightmargin=0.8em,
	framexrightmargin=0.5em,
	frame=single,						% Single frame around code
	frameround=ffff,
	rulecolor=\color{darkgray},			% Color of the frame
	keywordstyle=\color[rgb]{0.133,0.133,0.6},
	commentstyle=\color[rgb]{0.133,0.545,0.133},
	stringstyle=\color[rgb]{0.627,0.126,0.941},
	aboveskip=1.5em,
}
\ownlstlanguages


%%%%%%%%%% Create new save command %%%%%%%%%%

\makeatletter

% \savecode[OPTIONAL: e.g. <morekeywords=type>]{<ID>}{<language>}{<description>}{<file_name>}

% -> OPTIONAL to add any special settings (see https://texdoc.org/serve/listings/0)
% language: e.g. Java, C++, Python, ...
% description: caption of the code-block
% file_name: in the folder /content/non_tex_files/code
\newread\lstread % create read for counting the lines in a file
\newcommand\savecode[5][]{%
    \@namedef{code@#2}{{
		% Count the lines in the file to adjust the spacing of the code-block based on the result
		\newcount\linecnt
		\openin\lstread=../content/non_tex_files/code/#5
		\@whilesw\unless\ifeof\lstread\fi{%
			\advance\linecnt by \@ne
			\readline\lstread t\expandafter o\csname line-\number\linecnt\endcsname
		}
		\advance\linecnt by -1 % correct by one
		\newcommand{\amndig}{\fpeval{ ceil( ln(\linecnt+1)/ln(10) ) }} % get amount of digits in number of lines
		\closein\lstread
		
		% Include the code
		\lstinputlisting[
			language={#3},
			caption={#4},
			label={code:#2},
			% position line numbers depending on the amount of lines in the file
			xleftmargin=\fpeval{1.8+\amndig*0.5}em,
			framexleftmargin=\fpeval{1.5+\amndig*0.5}em,
			#1% add any special settings here, if #1 not empty
		]{../content/non_tex_files/code/#5}
    }}
}

% \usecode{<ID>}
\newcommand\usecode[1]{%
    \@nameuse{code@#1}
}

\makeatother


%%%%%%%%%% Content %%%%%%%%%%

% !TEX root = ../../document/document.tex
% LTeX: enabled=false


% \savecode[OPTIONAL: e.g. <morekeywords=type>]{<ID>}{<language>}{<description>}{<file_name>}

% -> OPTIONAL to add any special settings (see https://texdoc.org/serve/listings/0)
% language: e.g. Java, C++, Python, ... (see: https://en.wikibooks.org/wiki/LaTeX/Source_Code_Listings)
% description: caption of the code-block
% file_name: in the folder /content/non_tex_files/code


% chapter-01
\savecode{Beispiel_Code-2}{C++}{Beispiel: indirektes einfügen von Code über externe Datei}{chapter-01/hello-world.cpp}

