% !TEX root = ../../document/document.tex
% LTeX: enabled=false


%%%%%%%%%% Settings %%%%%%%%%%

% Change spaces of tables
\setlength{\tabcolsep}{10pt}
\renewcommand{\arraystretch}{1.5}


%%%%%%%%%% Create new save command %%%%%%%%%%

\makeatletter

% \savetable{<ID>}{<position>}{<adjust_settings>}{<table_width>}{<description>}{<column_setting>}{<content>}

% ID: name to save the table under
% position: where to position the table e.g. ht (here, top)
% adjust_settings: any settings for \begin{adjustbox} e.g. rotate=90, right
% table_width: width of the table e.g. \textwidth
% description: caption to show below the table
% column_setting: any column settings for \begin{tabular} e.g. | C | C | C |
% content: the content of the table
\newcommand\savetable[7]{%
    \@namedef{tab@#1}{{
        \begin{table}[#2]
            \centering
            \begin{adjustbox}{#3}
                \begin{tabularx}{#4}{#6}
                    #7
                \end{tabularx}
            \end{adjustbox}
            \caption{#5}
            \label{tab:#1}
        \end{table}
    }}
}

% \usetable{<ID>}
\newcommand\usetable[1]{%
    \@nameuse{tab@#1}
}

\makeatother


%%%%%%%%%% Additionals Commands %%%%%%%%%%

\newcolumntype{L}{>{\raggedright\arraybackslash}X}
\newcolumntype{C}{>{\centering\arraybackslash}X}
\newcolumntype{R}{>{\raggedleft\arraybackslash}X}


%%%%%%%%%% Content %%%%%%%%%%

% !TEX root = ../../document/document.tex
% LTeX: enabled=false


% \savetable{<ID>}{<position>}{<adjust_settings>}{<table_width>}{<description>}{<column_setting>}{<content>}

% ID: name to save the table under
% position: where to position the table e.g. ht (here, top)
% adjust_settings: any settings for \begin{adjustbox} e.g. rotate=90, right
% table_width: width of the table e.g. \textwidth
% description: caption to show below the table
% column_setting: any column settings for \begin{tabular} e.g. | C | C | C |
% content: the content of the table


% Define own column style j with fixed size and centered aligning for column 2 in this case
\newcolumntype{j}{>{\centering\arraybackslash\hsize=0.18\textwidth}X}
\savetable{bsp_normal}{!ht}{rotate=0, center}{\textwidth}{Beispieltabelle}
% Define columnstyles and vertical lines for the table and its content
{
    % Use a pipe | to create a vertical line with the default line width. 
    % Use !{\vrule width 3pt} instead, to create a vertical line with a custom line width of 3pt.
    | >{\hsize=0.2\textwidth}X % use default column type X with own size for column 1
    !{\vrule width 3pt} j
    | C
    | C
    | R
    | L
    |
}{
    \hline
    \rowcolor{lightgray}
        \textbf{Spalte 1} &
        \textbf{Spalte 2 (\%)} &
        \multicolumn{2}{|c|}{\textbf{Spalte 3}} & 
        \multicolumn{2}{|c|}{\textbf{Spalte 4}}
    % Use \hline to create a horizontal line with the default line height.
    % Use \noalign{\hrule height 3pt} to create a horizontal line with a custom line height.
    \\\noalign{\hrule height 3pt}
    Längerer Text mit automatischem Zeilenumbruch & 10 & 111 & 0,1 & Dies & 100
    \\\hline
    B & 20 & 222 & 0,2 & ist & 200
    \\\hline
    \multirow{2}{*}{C} & 30 & 333 & 0,3 & ein & 300
    \\\cline{2-6}
     & 40 & 444 & 0,4 & Beispiel & 400
    \\\hline
        \textbf{Summe} &
        \textbf{100} &
        \textbf{1110} &
        \textbf{1} & 
         &
        \textbf{1000}
    \\\hline
        \multicolumn{4}{|c|}{\textbf{Letzte}} &
        \multicolumn{2}{|c|}{\textbf{Reihe}}
    \\\hline
    Numbers & 1 & 2 & 3 & 4 & 5
    \\\hline
}

\savetable{bsp_rotiert}{hb}{rotate=90, right}{0.28\textwidth}{Rotierte Tabelle}
% Same size for all columns 
{| C | C | C |}
{
    \hline
    A & 10 & test
    \\\hline
    B & 20 & test
    \\\hline
}

